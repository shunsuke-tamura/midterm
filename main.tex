\documentclass[twocolumn,a4j]{jarticle}

%%
%%
%% 以下のスタイル情報は変更しない
\setlength{\textheight}{277mm}
\setlength{\textwidth}{180mm}
\setlength{\topmargin}{-35mm}
\setlength{\oddsidemargin}{-9mm}

\usepackage{secdot}
\usepackage{listings, jvlisting}
\sectiondot{subsection}

%%
%% 
%% 論文タイトルを記入
\title{
  オリジナル画像のダウンロードが困難な\\画像掲載Webアプリケーションの作成
}   

%%
%% 
%% 氏名、学籍番号、指導教員を記入
\author{
  佐賀大学 理工学部 理工学科 知能情報システム工学コース\\
   発表者:田村 駿典 (20238264)\\
  指導教員:松前 進 教授
}


\begin{document}
\date{\empty}
\maketitle
\thispagestyle{empty}

%%
%%
%% section や subsection を必要に応じて用い、本文を記述する。
\section{はじめに}\label{sec:sec1}

昨今、Midjourney、DALL・E2、Stable Diffusionといった2次元画像を生成することのできるAIの発展が目覚ましい。\par
そんな中、日本ではAIの著作権に関する法整備が十分でないため、プロのイラストレーターがインターネット上にアップロードした作品を無断でAIの学習に利用されてしまうという事態に陥っている\cite{nhk}。
現在、こう言った無断使用を抑制するためには、イラストをアップロードする際にイラストに対して透かしを入れたり、解像度を下げたりというような製作者にとって不利益になるような対策をするしかない。\par
そこで、オリジナル画像のダウンロードが極めて困難な画像アップロードWebアプリケーション(以下、本アプリ)を作成する。



\section{サービスの概要}
本アプリでは、\ref{sec:sec1} で述べた課題解決のために、以下の主要機能を提供することを目指す。
\begin{itemize}
  \item 画像アップロード機能
        \setlength{\parskip}{0cm}
        \begin{itemize}
          \item 画像を短い文章とともに本アプリに投稿することができる
        \end{itemize}
  \item 画像閲覧機能
        \begin{itemize}
          \item アップロードされた画像を閲覧することができる
          \item 一覧性をを重視して、サムネイルを一覧で表示する一覧ページと、そのページで選択された画像のオリジナル画像を表示する詳細ページを提供する
          \item 詳細ページで表示されているオリジナル画像は、詳細ページからに対して、スクリプトの実行や、ブラウザの開発者ツールの使用でオリジナル画像のダウンロードが困難であるように実装する
        \end{itemize}
\end{itemize}


\section{使用技術}
本アプリは、以下の技術を使用して作成する予定である。
\begin{description}
  \item[フロントエンド]\mbox{}\\
  Next.jsを使用する。\\
  現在、ブラウザから任意のJavaScriptを実行できてしまうという都合上、ブラウザへオリジナル画像が送信されてしまっては簡単にデータをダウンロードできてしまうため、ブラウザへオリジナル画像が送信されないことが最低条件となってくる。\\
  そこで、Next.jsのSSR機能を使用し、サーバサイドでオリジナル画像を使用する処理を済ませる。
  \item[バックエンド]\mbox{}\\
  Golangを使用する。\\
  画像を扱うWebアプリケーションであるため、高いパフォーマンスが要求されることが予想される。\\
  そこで、高速なWebアプリケーションの作成に適したGolang\cite{go} をバックエンドに使用する。
\end{description}


\section{進捗状況と今後}
前期では、本課題を解決するための方法を探るため、実際に小さなWebブラウザを作成することで、ブラウザがコンテンツを表示する仕組みを理解した。\par
その学習の結果、画像を表示する際に元となるデータをJavaScriptがアクセスできる場所に保持する必要があるため、100\% オリジナル画像のダウンロードを防ぐことはできない。という結論に至った。\par
そこで、なるべくオリジナル画像のダウンロードを困難にするために、どのような手段があるかを模索している。現状最良だと思われる手段は、大きさが1pxしかない\lstinline!<p>!にオリジナル画像の該当ピクセルと同じ色をCSSで設定し、\lstinline|<p>|の集合で画像を表現するという方法だ。\par
今後は、この方法を発展させ、なるべくオリジナル画像のダウンロードが困難になる方法を考え実装していく予定だ。


%%
%%
%% 参考文献を記述する。
\begin{thebibliography}{9}\setlength{\itemsep}{-2pt}
  \bibitem{nhk} NHK 画像生成AI“クリエーターの権利脅かされる”法整備など提言(https://www3.nhk.or.jp/news/htm\\l/20230427/k10014051061000.html)
  \bibitem{go} The Go Programming Language(https://go.dev/)
\end{thebibliography}
\end{document}