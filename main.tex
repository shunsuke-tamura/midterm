\documentclass[twocolumn,a4j]{jarticle}

%%
%%
%% 以下のスタイル情報は変更しない
\setlength{\textheight}{277mm}
\setlength{\textwidth}{180mm}
\setlength{\topmargin}{-35mm}
\setlength{\oddsidemargin}{-9mm}

\usepackage{secdot}
\sectiondot{subsection}

%%
%% 
%% 論文タイトルを記入
\title{
  卒業論文または中間発表のタイトル
}   

%%
%% 
%% 氏名、学籍番号、指導教員を記入
\author{
  佐賀大学 理工学部 知能情報システム学科\\
   発表者:知能 情太郎 (1x233yyy)\\
  指導教員:情報 シス夫 准教授
}


\begin{document}
\date{\empty}
\maketitle
\thispagestyle{empty}

%%
%%
%% section や subsection を必要に応じて用い、本文を記述する。
\section{はじめに }

言語は日本語または英語とする.
A4用紙の片面のみを縦に使用する.
記述は横書きとし,鉛筆書きは不可とする.
上下の余白は10mm,左右の余白は15mmとする.
本文は2段組とする.
長さは1枚以内とし,ページ番号は打たない.
タイトル,発表者氏名,指導教員氏名,発表要旨を含むこと.



\section{章のタイトル}
2章の内容
\\
\\
\\
\\
\\
\\
\\
\\


\section{章のタイトル}
3章の内容
\\
\\
\\
\\
\\
\\
\\
\\

\section{章のタイトル}
4章の内容
\\
\\
\\
\\
\\
\\
\\
\\
\\
\\
\\
\\
\\
\\
\\
\\
\\
\\
\\
\\
\\
\\
\\
\\
\\
\\
\\
\\
\\
\\
\\
\\
\\
\\
\\

%%
%%
%% 参考文献を記述する。
\begin{thebibliography}{9}\setlength{\itemsep}{-2pt}
  \bibitem{yamgami} 山上一郎,山下二郎,``パラメトリック増幅器,'' 信学論(B),vol.J62-B,no.1,pp.20--27,Jan.\ 1979.
  \bibitem{rice} W.~Rice, A.C.~Wine, and B.D.~Grain, ``Diffusion of impurities during epitaxy.'' Proc.\ IEEE, vol.52, no.3, pp.284--290, March 1964.
  \bibitem{yamada} 山田太郎,移動通信,木村次郎(編),(社)電子情報通信学会,東京,1989.
  \bibitem{provan} D.~Provan, ``Request for comments 1234: Tunneling IPX traffic through IP networks,'' IETF, http://www.ietf.org/rfc/rfc1234.txt, June 1991.
\end{thebibliography}
\end{document}