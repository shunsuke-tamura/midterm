\documentclass[twocolumn,a4j]{jarticle}

%%
%%
%% 以下のスタイル情報は変更しない
\setlength{\textheight}{277mm}
\setlength{\textwidth}{180mm}
\setlength{\topmargin}{-35mm}
\setlength{\oddsidemargin}{-9mm}

\usepackage{secdot}
\usepackage{listings, jvlisting}
\sectiondot{subsection}

%%
%% 
%% 論文タイトルを記入
\title{
  オリジナル画像のダウンロードが困難な\\画像掲載Webアプリケーションの作成
}   

%%
%% 
%% 氏名,学籍番号,指導教員を記入
\author{
  佐賀大学 理工学部 理工学科 知能情報システム工学コース\\
   発表者:田村 駿典 (20238264)\\
  指導教員:松前 進 教授
}


\begin{document}
\date{\empty}
\maketitle
\thispagestyle{empty}

%%
%%
%% section や subsection を必要に応じて用い,本文を記述する.
\section{はじめに}\label{sec:sec1}

昨今,Midjourney,DALL・E2,Stable Diffusionに代表される2次元画像を生成することのできるAIの発展が目覚ましいが,日本ではAIの著作権に関する法整備が不十分である.故に,プロのイラストレーターがアップロードしたインターネット上の作品を無断でAIの学習に利用されることがよくある\cite{nhk}.
現在,このような無断使用を抑制するためには,イラストをアップロードする際にイラストに対して透かしを入れる・解像度を下げるといった製作者にとって不利益になるような対策をするしかない.\par
本研究では,上記の問題をオリジナル画像のダウンロードが極めて困難な画像アップロードWebアプリケーション(以下,本アプリ)を作成することで解決を試みる.



\section{サービスの概要}
本アプリでは,\ref{sec:sec1} で述べた課題解決のため,以下の主要機能を提供することを目指す.
\begin{itemize}
  \item 画像アップロード機能
        \setlength{\parskip}{0cm}
        \begin{itemize}
          \item 画像を短い文章とともに本アプリに投稿することができる
        \end{itemize}
  \item 画像閲覧機能
        \begin{itemize}
          \item アップロードされた画像を閲覧することができる
          \item 一覧性を重視して,サムネイルを表示する一覧ページ,そのページで選択された画像のオリジナル画像を表示する詳細ページを提供する
          \item 詳細ページで表示されているオリジナル画像は,スクリプトの実行,ブラウザの開発者ツールの使用でオリジナル画像のダウンロードが困難であるように実装する
        \end{itemize}
\end{itemize}


\section{使用技術}
本アプリは,以下の技術を使用して作成する予定である.
\begin{description}
  \item[フロントエンド]\mbox{}\\
  Next.jsを使用する.\\
  現在,ブラウザから任意のJavaScriptを実行できるため,ブラウザへオリジナル画像が送信されると簡単にデータをダウンロードできる,ブラウザへオリジナル画像が送信されないことが最低条件である.\\
  従って,Next.jsのSSR機能を使用し,サーバサイドでオリジナル画像を使用する処理を実行するする.
  \item[バックエンド]\mbox{}\\
  Golangを使用する.\\
  画像を扱うWebアプリケーションであるため,高いパフォーマンスが要求される.\\
  従って,高速なWebアプリケーションの作成に適したGolang\cite{go} をバックエンドに使用する.
\end{description}


\section{進捗状況と今後}
前期では,本課題を解決するための方法を探るため,実際に小さなWebブラウザを作成することで,ブラウザがコンテンツを表示する仕組みを理解した.\par
その学習の結果,画像を表示する際に元となるデータをJavaScriptがアクセスできる場所に保持することが必要であるため,オリジナル画像のダウンロードを完全に防ぐことはできない,という結論に至った.\par
従って,可能な限りオリジナル画像のダウンロードを困難にするために,どのような手段があるかを模索している.現在検討中の手段は,大きさが1pxである\lstinline!<p>!にオリジナル画像の該当ピクセルと同じ色をCSSで設定し,\lstinline|<p>|の集合で画像を表現するという方法である.\par
今後は,この方法を発展させ,オリジナル画像のダウンロードがより困難になる方法を考え実装していく予定である.


%%
%%
%% 参考文献を記述する.
\begin{thebibliography}{9}\setlength{\itemsep}{-2pt}
  \bibitem{nhk} NHK 画像生成AI“クリエーターの権利脅かされる”法整備など提言(https://www3.nhk.or.jp/news/htm\\l/20230427/k10014051061000.html) 閲覧日 2023/08/31
  \bibitem{go} The Go Programming Language(https://go.dev/) 閲覧日 2023/08/31
\end{thebibliography}
\end{document}